\documentclass[11pt]{article}
\usepackage{amsmath,euscript}
\usepackage{amssymb}
\usepackage{amsthm}
\usepackage{enumerate}
\usepackage{amsfonts}
\usepackage{amscd}

\usepackage{courier}
\usepackage{verbatim}
\usepackage{fancyvrb}

\usepackage{tikz} 

\begin{document}

\title{Project 2}
\date{}
\author{Archie Mead}
%don't mess with my circles

\maketitle 

\section*{The Idea}

This method interpolates between points by drawing arcs of \emph{circles} that connect groups of three of them. The subsets of points \((x_{2i}, y_{2i})\), \((x_{2i+1}, y_{2i+1})\), \((x_{2i+2}, y_{2i+2})\) each have a unique circle (or line) that intersects all three of them. We define the function:

$$S_{2i}(t) = 
\left[
\begin{array}{c}
a_{2i} + r_{2i} \cos(\theta_{2i} + \phi_{2i} t) \\
b_{2i} + r_{2i} \sin(\theta_{2i} + \phi_{2i} t) \\
\end{array}
\right]
$$

Where \((a_i, b_i)\) is the center of this circle, \(r_i\) is the radius, and \(\theta_i, \phi_i\) are angles such that \(S_{2i}(0) = \langle x_{2i}, y_{2i}\rangle\) and \(S_{2i}(1) = \langle x_{2i+2}, y_{2i+2} \rangle\). For some value \(t_{2i+1} \in [0,1]\), we have \(S_{2i}(t_{2i+1}) = \langle x_{2i+1}, y_{2i+1}\rangle\).

If the three points happen to be collinear, we instead use a line:
$$S_{2i}(t) = 
\left[
\begin{array}{c}
(1 - t) x_{2i} + t x_{2i+2} \\
(1 - t) y_{2i} + t y_{2i+2} \\
\end{array}
\right]
$$

If there is an even number of points, we also use a line to connect the last two. 

This interpolation connects all points. There is usually a corner at every other point. If the figure of points is rotates, the interpolation will consistently rotate with it. The points need not be ordered left-to-right, and the path can bend in any direction. 


\section*{The Computation}

Let each \(\vec x_i = \langle x_i, y_i \rangle\), and each \(\vec a_{2i} = \langle a_{2i}, b_{2i} \rangle\). 

The circle center \(\vec a_{2i}\) must be equidistant from \(\vec x_{2i}, \vec x_{2i+1}, \vec x_{2i+2}\). 

$$ || \vec a_{2i} - \vec x_{2i} ||^2 = 
|| \vec a_{2i} - \vec x_{2i+1} ||^2 = 
|| \vec a_{2i} - \vec x_{2i+2} ||^2 $$

$$ 
|| \vec a_{2i} ||^2 - 2 \vec a_{2i} \cdot \vec x_{2i} + || \vec x_{2i} ||^2 = 
$$
$$ 
|| \vec a_{2i} ||^2 - 2 \vec a_{2i} \cdot \vec x_{2i+1} + || \vec x_{2i+1} ||^2 = 
$$
$$ 
|| \vec a_{2i} ||^2 - 2 \vec a_{2i} \cdot \vec x_{2i+2} + || \vec x_{2i+2} ||^2
$$

Use \(\Delta\) notation to refer to forward differences. For example, \(\Delta x_{2i} = x_{2i+1} - x_{2i}\). 

$$ 2 \vec a_{2i} \cdot \Delta \vec x_{2i} = \Delta (||\vec x_{2i} ||^2)
$$
$$ 2 \vec a_{2i} \cdot \Delta \vec x_{2i+1} = \Delta (||\vec x_{2i+1} ||^2)
$$

$$
2
\left[
\begin{array}{cc}
\Delta x_{2i} & \Delta y_{2i} \\
\Delta x_{2i+1} & \Delta y_{2i+1} \\
\end{array}
\right]
\left[
\begin{array}{c}
a_{2i} \\ b_{2i}
\end{array}
\right]
=
\left[
\begin{array}{c}
\Delta (||\vec x_{2i} ||^2) \\
\Delta (||\vec x_{2i+1} ||^2) \\
\end{array}
\right]
$$

If the determinant of the matrix on the left is zero, the points are collinear, and we use the other formula. 

Otherwise, once \(\vec a_{2i}\) is calculated, we get:

$$ r_{2i} = || \vec x_{2i} - \vec a_{2i} || $$

To work with the angles, we use the \({\text atan2}(x,y)\) function, which determines the angle needed to draw segment in the direction \(\langle x, y\rangle\). (This matches \(\arctan(y/x)\) when \(x > 0\), but is off by \(\pm \pi\) otherwise.)

$$ \theta_{2i} = \text{atan2}(y_{2i} - b_{2i}, x_{2i} - a_{2i}) $$

We want to make is so that $S_{2i}(t)$ moves in the correct direction to pass through \(\vec x_{2i}, \vec x_{2i+1}, \vec x_{2i+2}\) in that order. We set: 
$$ \alpha = \text{atan2}(y_{2i+1} - b_{2i}, x_{2i+1} - a_{2i}) - \theta_{2i} $$
$$ \beta = \text{atan2}(y_{2i+2} - b_{2i}, x_{2i+2} - a_{2i}) - \theta_{2i}  - \alpha$$
We want $\alpha, \beta$ to be the radians travels from each point to the next. The should be in the same direction, and with $\alpha + \beta \in [-2\pi, 2\pi]$.
\begin{itemize}
	\item If $\alpha, \beta$ have the same sign, and $|\alpha + \beta| \le 2\pi$, then we leave them as-is.
	\item If $\alpha, \beta > 0$ , but $|\alpha + \beta| > 2\pi$, then we adjust them both by \(-2\pi\). If instead $\alpha, \beta < 0$, we adjust them by \(+ 2\pi\).
	\item If $\alpha, \beta$ have different signs, then we take the one closer to zero and adjust it by $\pm 2\pi$, using the opposite of its current sign (sending it to the other side of zero).
\end{itemize}

After this step, we set $\phi_{2i} = \alpha + \beta$.



\section*{The Results}


% tikz is a widely-used package for LaTeX with a LOT of flexibilty.
% These diagrams use small circles to mark the points.
% The coordinates of the points and lines are in parentheses
% fbox gives it a frame.


Graph 1:
\begin{center}
	\fbox{
		\begin{tikzpicture}
		\draw[fill] (0,2) circle [radius=0.05];
		\draw[fill] (2,1) circle [radius=0.05];
		\draw[fill] (3,0) circle [radius=0.05];
		\draw[fill] (5,1) circle [radius=0.05];
		\draw[fill] (6,2) circle [radius=0.05];
		\draw[fill] (9,0) circle [radius=0.05];
		
		\draw[blue,variable=\t,domain=0:1,samples=100]
		plot ({-1.5 + 5.7009 * cos(180/pi*(1.3045 + -0.6435*\t))},
		{-3.5 + 5.7009 * sin(180/pi*(1.3045 + -0.6435*\t))});
		\draw[blue,variable=\t,domain=0:1,samples=100]
		plot ({1.5 + 5.7009 * cos(180/pi*(-1.3045 + 0.6435*\t))},
		{5.5 + 5.7009 * sin(180/pi*(-1.3045 + 0.6435*\t))});
		\draw[blue,variable=\t,domain=0:1,samples=100]
		plot ({6 + 3*\t},
		{2 - 2*\t});
		\end{tikzpicture}
	}
\end{center}

Graph 2:
\begin{center}
	\fbox{
		\begin{tikzpicture}
		\draw[fill] (0,0) circle [radius=0.05];
		\draw[fill] (1,1) circle [radius=0.05];
		\draw[fill] (2,2) circle [radius=0.05];
		\draw[fill] (4,4) circle [radius=0.05];
		\draw[fill] (5,3) circle [radius=0.05];
		\draw[fill] (6,1) circle [radius=0.05];
		
		\draw[blue,variable=\t,domain=0:1,samples=100]
		plot ({0 + 2*\t},
		{0 + 2*\t});
		\draw[blue,variable=\t,domain=0:1,samples=100]
		plot ({3.5 + 1.5811 * cos(180/pi*(-2.8198 - pi*\t))},
		{2.5 + 1.5811 * sin(180/pi*(-2.8198 - pi*\t))});
		\draw[blue,variable=\t,domain=0:1,samples=100]
		plot ({5 + 1*\t},
		{3 - 2*\t});
		\end{tikzpicture}
	}
\end{center}

Graph 3:
\begin{center}
	\fbox{
		\begin{tikzpicture}
		\draw[fill] (0,2) circle [radius=0.05];
		\draw[fill] (1,1) circle [radius=0.05];
		\draw[fill] (2,2) circle [radius=0.05];
		\draw[fill] (4,5) circle [radius=0.05];
		\draw[fill] (5,6) circle [radius=0.05];
		\draw[fill] (9,5) circle [radius=0.05];
		\draw[fill] (10,5) circle [radius=0.05];
		
		\draw[blue,variable=\t,domain=0:1,samples=100]
		plot ({1.0 + 1.0 * cos(180/pi*(pi + pi*\t))},
		{2.0 + 1.0 * sin(180/pi*(pi + pi*\t))});
		\draw[blue,variable=\t,domain=0:1,samples=100]
		plot ({13.5 + 12.7475 * cos(180/pi*(2.6955 - 0.3948*\t))},
		{-3.5 + 12.7475 * sin(180/pi*(2.6955- 0.3948*\t))});
		\draw[blue,variable=\t,domain=0:1,samples=100]
		plot ({9.5 + 10.5119 * cos(180/pi*(-2.0132 + 0.4900*\t))},
		{15.5 + 10.5119 * sin(180/pi*(-2.0132 + 0.4900*\t))});
		\end{tikzpicture}
	}
\end{center}

Graph 4:
\begin{center}
	\fbox{
		\begin{tikzpicture}
		\draw[fill] (0,0) circle [radius=0.05];
		\draw[fill] (1,0) circle [radius=0.05];
		\draw[fill] (2,0) circle [radius=0.05];
		\draw[fill] (3,0) circle [radius=0.05];
		\draw[fill] (4,2) circle [radius=0.05];
		\draw[fill] (5,0) circle [radius=0.05];
		\draw[fill] (6,0) circle [radius=0.05];
		\draw[fill] (7,0) circle [radius=0.05];
		
		\draw[blue,variable=\t,domain=0:1,samples=100]
		plot ({0 + 2*\t},
		{0 + 0*\t});
		\draw[blue,variable=\t,domain=0:1,samples=100]
		plot ({2.5 + 1.5811 * cos(180/pi*(-1.8925 + 2.2143*\t))},
		{1.5 + 1.5811 * sin(180/pi*(-1.8925 + 2.2143*\t))});
		\draw[blue,variable=\t,domain=0:1,samples=100]
		plot ({5.5 + 1.5811 * cos(180/pi*(2.8198 + 2.2143*\t))},
		{1.5 + 1.5811 * sin(180/pi*(2.8198 + 2.2143*\t))});
		\draw[blue,variable=\t,domain=0:1,samples=100]
		plot ({6 + 1*\t},
		{0 + 0*\t});
		\end{tikzpicture}
	}
\end{center}

Graph 5:
\begin{center}
	\fbox{
		\begin{tikzpicture}
		\draw[fill] (0,5) circle [radius=0.05];
		\draw[fill] (1,2) circle [radius=0.05];
		\draw[fill] (2,1) circle [radius=0.05];
		\draw[fill] (4,0) circle [radius=0.05];
		\draw[fill] (6,1) circle [radius=0.05];
		\draw[fill] (7,3) circle [radius=0.05];
		\draw[fill] (9,4) circle [radius=0.05];
		\draw[fill] (10,2) circle [radius=0.05];
		
		\draw[blue,variable=\t,domain=0:1,samples=100]
		plot ({5 + 5 * cos(180/pi*(pi + 0.9273*\t))},
		{5 + 5 * sin(180/pi*(pi + 0.9273*\t))});
		\draw[blue,variable=\t,domain=0:1,samples=100]
		plot ({4 + 2.5 * cos(180/pi*(-2.4981 + 1.8546*\t))},
		{2.5 + 2.5 * sin(180/pi*(-2.4981 + 1.8546*\t))});
		\draw[blue,variable=\t,domain=0:1,samples=100]
		plot ({9.5 + 3.5355 * cos(180/pi*(2.9997 + -1.287*\t))},
		{0.5 + 3.5355 * sin(180/pi*(2.9997 + -1.287*\t))});
		\draw[blue,variable=\t,domain=0:1,samples=100]
		plot ({9 + 1*\t},
		{4 - 2*\t});
		\end{tikzpicture}
	}
\end{center}

\section*{Possible Issues}

This method usually makes sharp corners at every other point, which may not be desirable.

Inserting a point anywhere except the right end will change the even/odd designation of every point to its right, which may alter the picture drastically. 

\section*{Possible Extensions}

We believe that it is possible to interpolate arcs that lie tangent on the points they intersect (no corners) if we set each to connect two consecutive points instead of three.

It might also be possible to make the current arcs all tangent if we permit ourselves to use \emph{ellipses} as well as circles, or other conic sections. 


\end{document}